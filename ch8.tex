\documentclass{beamer}
\usetheme[hideothersubsections]{FSUTheme}

\begin{document}

\title{\emph{Why We Should Put the Death Penalty to Rest}}
\author{Stephen Nathanson} %
\institute{FSU}
\date{\today}

\section{Thesis}

\begin{frame}
    \frametitle{Thesis}
    \framesubtitle{}

The death penalty is inconsistent with the importance most people place 
on justice, and the respect most people have for human life.

\end{frame}

\section{}    

\begin{frame}
    \frametitle{A logical point}
    \framesubtitle{}

\begin{enumerate}
\item People the ideals of human life and the importance of justice
\item These ideals are inconsistent with the death penalty \\
\rule {1in}{.5pt}
\item $\therefore$ Holding both contradicts these peoples' values

\pause
\item You might deny holding the ideals of human life and the importance 
of justice
\item But that would ruin the credibility of your argument, and 
undermine your reasons for supporting the death penalty.

\end{enumerate}


\end{frame}

\section{Deterrence}

\begin{frame}
    \frametitle{Deterrence}
    \framesubtitle{}
\begin{enumerate}
\item The threat of execution is a greater deterrent than lesser 
punishments and will lead to fewer murders
\item If the death penalty is uniquely effective at deterring murder, 
then it is justified


\end{enumerate}



\end{frame}


\section{Retribution}

\begin{frame}
    \frametitle{Retribution}
    \framesubtitle{}
\begin{enumerate}
\item The only just punishment for capital crimes is capital punishment
\item Anything less than capital punishment is morally unacceptable
\item The punishment should equal the crime.



\end{enumerate}



\end{frame}


\section{Problems}

\begin{frame}
    \frametitle{Problems with Deterrence}
    \framesubtitle{}
Underlying Moral Principle -- \\
-- If a punishment deters more murders and thus saves more innocent 
lives, it is justifiable.\\

\begin{enumerate}
\item We can imagine punishments that have greater deterrent value, but 
would be wrong to inflict
\item So, even if it is the best deterrent, it may still be wrong
\item So, the argument does not justify the death penalty 


\end{enumerate}



\end{frame}


\section{}

\begin{frame}
    \frametitle{Problems with Retribution}
    \framesubtitle{}

Underlying moral principle -- \\
-- We should do to the criminal what he/she has done to the victim

\begin{enumerate}
\item The ``eye for an eye'' principle is obviously defective
\item Not only is it defective, but \emph{you don't even believe it}
\begin{enumerate}
\item It requires barbaric responses to barbaric crimes
\item It conflicts with our beliefs about the justification of punishment
\item It is often ambiguous
\end{enumerate}

\end{enumerate}



\end{frame}


\section{}

\begin{frame}
    \frametitle{Problems with Death Penalty \emph{in Practice}}
    \framesubtitle{}

\begin{enumerate}
\item Even if Capital punishment really was the best deterrent against 
murder, and even if ``eye for an eye'' theory was an adequate procedure 
for determining proper punishments, it still doesn't follow that 
\emph{we} should adopt the death penalty.
\item There is a distinction between `justified in theory' and 
`justified in practice'
\item e.g., Even if the above were true (i.e., it were justified in 
theory), if 
half the people who were executed were innocent, and if we knew it, we 
could not justify it in practice
\pause

\end{enumerate}

What do were learn from this example?

\end{frame}


\section{Inconsistency of the practice and the values}

\begin{frame}
    \frametitle{The Death Penalty is Inconsistent with the Value of 
Justice}
    \framesubtitle{}
\begin{enumerate}
\item Capital Punishment is administered by real people in 
actual societies, and kills real people
\item If there is an inconsistency, it is between the values of the 
supporters of the death penalty, and the practice of capital punishment
\item If you consider your own values, you'll see that they're violated 
by the institution of capital punishment

\end{enumerate}



\end{frame}


\section{}

\begin{frame}
    \frametitle{A just system}
    \framesubtitle{}
\begin{enumerate}
\item A system is just if:
\begin{enumerate}
\item It separates the guilty from the innocent, and
\item It separates those who deserve to die from those who deserve some 
lesser punishment


\end{enumerate}
\end{enumerate}

A system that fails either of these criteria is not just \\

In theory, the death penalty is assigned to those who commit a specific 
crime with exceptional terribleness.  In fact, other factors play a 
role, too.


\end{frame}


\section{Biased decision procedure}

\begin{frame}
    \frametitle{Race}
    \framesubtitle{}
\begin{enumerate}
\item 83\% of people executed between 1976 and 1996 were found guilty of 
killing a white person
\item Only 1\% of people executed between 1976 and 1996 were white 
persons found guilty of killing a black person


\end{enumerate}



\end{frame}


\section{}

\begin{frame}
    \frametitle{Socio-Economic Status}
    \framesubtitle{}
\begin{enumerate}
\item In GA, a poor defendant is 2.3 times more likely to receive the 
death penalty


\end{enumerate}



\end{frame}


\section{}

\begin{frame}
    \frametitle{Legal Representation}
    \framesubtitle{}
\begin{enumerate}
\item Defendants using court appointed representation were 2.6 times 
more likely to receive the death penalty


\end{enumerate}



\end{frame}


\section{Inconsistent with respect to human life}

\begin{frame}
    \frametitle{The death penalty is inconsistent with a commitment to 
the value we place on human life}
    \framesubtitle{}
\begin{enumerate}
\item There is an inconsistency between affirming the value of human 
life and tolerating the current level of legal representation for 
people who 
face the possibility of death.
\item Our system may lead to two sorts of mistaken judgments:
\begin{enumerate}
\item A criminal may get a more severe sentence because of race/social 
status/economic status.
\item Poor legal representation may result in an innocent person being 
sentenced to death.




\end{enumerate}
\end{enumerate}



\end{frame}


\section{}

\begin{frame}
    \frametitle{Illinois Moratorium on executions}
    \framesubtitle{}
\begin{enumerate}
\item Between 1977-2000 12 innocent people were executed, and 13 
completely innocent people 
were released from death row
\item In some cases, police use coercive measures to extract confessions
\item In at least 46 cases, murder convictions were based on testimony 
of other criminals whose sentence is reduced for testifying.


\end{enumerate}
These facts are even worse for executions, since the death penalty makes 
corrections of errors irreversible


\end{frame}


\section{}

\begin{frame}
    \frametitle{Respect for Human Life}
    \framesubtitle{}
\begin{enumerate}
\item Herrera v. Collins -- 60 day limits on new evidence are not 
consistent with a commitment to respecting human life
\item These practices reflect the facts that:
\begin{enumerate}
\item Legal bureaucracy desires to bring time consuming appeals to a 
halt
\item Officials do not want to be seen as incompetent
\item Citizens want lower taxes more than they want to pay for competent 
lawyers for people charged with murder
\item It is easier to respect the value of human life in words than in 
deeds

\end{enumerate}
\end{enumerate}



\end{frame}


\section{}

\begin{frame}
    \frametitle{}
    \framesubtitle{}
\begin{enumerate}
\item 


\end{enumerate}



\end{frame}


\section{}

\begin{frame}
    \frametitle{}
    \framesubtitle{}



\end{frame}


\section{}

\begin{frame}
    \frametitle{}
    \framesubtitle{}



\end{frame}


\section{}

\begin{frame}
    \frametitle{}
    \framesubtitle{}



\end{frame}


\section{}

\begin{frame}
    \frametitle{}
    \framesubtitle{}



\end{frame}


\section{}

\begin{frame}
    \frametitle{}
    \framesubtitle{}



\end{frame}


\section{}

\begin{frame}
    \frametitle{}
    \framesubtitle{}



\end{frame}


\section{}

\begin{frame}
    \frametitle{}
    \framesubtitle{}



\end{frame}


\section{}

\begin{frame}
    \frametitle{}
    \framesubtitle{}



\end{frame}


\section{}

\begin{frame}
    \frametitle{}
    \framesubtitle{}



\end{frame}


\section{}

\begin{frame}
    \frametitle{}
    \framesubtitle{}



\end{frame}


\section{}

\begin{frame}
    \frametitle{}
    \framesubtitle{}



\end{frame}


\section{}

\begin{frame}
    \frametitle{}
    \framesubtitle{}



\end{frame}

    
    
\end{document}
